
\chapter*{Introduction}

Le challenge PLAsTiCC (Photometric LSST Astronomical Time Series Classification Challenge) est une compétition de sciences des données qui consiste à classer des données astronomiques simulées. Les méthodes développées au cours de ce challenge seront par la suite adaptées à des observations réelles effectuées par le LSST (Large Synoptic Survey Telescope) qui commencera ses mesures en 2022 pour une durée de 10 ans. Le LSST pourrait révolutionner notre compréhension de la mécanique céleste en découvrant et en suivant des millions d’objets astronomiques.
\newline
Ainsi, dans le cadre du projet d’étude de cas décisionnelle, nous avons constitué une
équipe pour participer à ce challenge et essayer de développer un modèle permettant de
résoudre la problématique du challenge.
\newline
Dans ce projet, nous nous focalisons sur les éléments dont la luminosité évolue au cours du temps. Nous chercherons à évaluer à quel point nous pouvons classer des objets astronomiques dont la luminosité varie au cours du temps à partir de séries temporelles du LSST simulées en prenant en compte les challenges liés à la non-représentativité de ces données. Dans un premier temps, nous détaillerons le contexte et mettrons en perspective notre objectif en abordant l’existant et l’état de l’art dans le domaine. Nous poursuivrons par un focus sur la méthode employée pour traiter le sujet. Nous finirons par une présentation et une analyse des résultats obtenus. 


